\documentclass{letter}

\usepackage[addressright]{vandyletter}


\name{Your Name}
\telephone{Phone: 615 555 5555}
\email{your.name@vanderbilt.edu}
\web{http://your.address.html}
% Leave empty if you don't want an item,e.g.  \email{}
% Use \address{row 1 \\ row 2 \\ row 3, etc...} if you want to override the default

\begin{document}


\begin{letter}{
To: Economics Professors\\
Calhoun Hall \\
Vanderbilt University Econ Dept}

\opening{
Dear Colleagues,
}


Good news! I have managed to create a LaTeX package reproducing Vanderbilt
Department of Economics's letterhead. It works under the LaTeX standard
letter style. The only difference with the layout of the official letterhead
is the addition of the words "Department of Economics" to the department
address in the footer.

I haven't tested all the options, please let me know if anything doesn't
work as expected at my email andrea.moro@vanderbilt.edu. The package can be
used with scientific word or any other pdfLaTeX compiler (see specific
scientific word instructions below). The package distribution contains 5
files: vandyletter.sty, cas04a.pdf, cas04acolor.pdf, vandyletter.tex and
vandyletter.pdf. The first three files are necessary. File vandyletter.tex
is this document, which you can use as a template, and vandyletter.pdf its
pdf version. \bigskip

\textbf{Style options}

The package comes with the following options (see below how to set the
options in Scientific Word)/ The default produces a color logo with address on the left.

\textbf{[no-logo]} for printing on pre-printed letterhead\vspace{.8ex} \newline 
\textbf{[addressright]} places your address to the right\vspace{.8ex} \newline 
\textbf{[color-logo]} prints vb's logo with the standard gold-v acorn \vspace{.8ex} \newline 
\textbf{[bw-logo]} prints vb's logo in black\vspace{.8ex} \newline 
\textbf{[britdate]} prints date as day-month-year\vspace{.8ex} \newline 
\textbf{[usesignature]} places the signature picture signature.jpg before your name
at the end (you need to create and place this file in the same directory
where your document is)\vspace{.8ex} \newline 
\textbf{[nofoot]} doesn't print the department's address in the footer of the first
page (all pages except the first will print a page number) \vspace{.8ex} \newline 
\textbf{[printenvelope]}
prints the address for an envelope \vspace{.8ex} \newline 
\textbf{[printlabel]} prints the address on a separate page \vspace{.8ex} \newline 


In the preamble of your file, you can specify the following (please see
below for how to do this in Scientific Word):\newline
\newline
\texttt{\textbackslash name\{Sender's name\}} (required) \newline
\texttt{\textbackslash telephone\{phone number\}} \newline
\texttt{\textbackslash email\{email address\}}\newline
\texttt{\textbackslash address\{address row 1 \textbackslash \textbackslash 
address row 2 \textbackslash \textbackslash  address row 3\}}

The declaration \textbackslash address, if present, supercedes 
\textbackslash \textbackslash name \textbackslash \textbackslash 
\\telephone \textbackslash \textbackslash email and can be used to change the
default to any content you like.\bigskip

\textbf{How to use this package with Scientific word}

If you use scientific word, place the three files anywyere under the folder 
\texttt{tcitex/tex/latex }of
your scientific word folder (usually \texttt{c:/sw55} or \texttt{c:/swp55})
(you may create a special folder if you wish). Place this file
(vanderbilt.tex) in the directory where you save your letters, open it and
modify it.

1. The first thing you want to do after opening this file is modify the
preamble with your name and numbers (make sure you don't remove any of the
braces,or your letter will not compile). The preamble can be found under the
menu \emph{typeset/preamble}. That's where you specify your name and
numbers. If you like a different format, write it in the preamble using 
\texttt{\textbackslash address\{row 1 
 \textbackslash \textbackslash  row 2 \textbackslash \textbackslash 
row 3\}}. Note that each row is separated by LaTeX's newline command, a
double backslash.

2. Next, click on the buttons "begin", "opening", and "closing" of this
template and change to your liking. Don't remove the open and close braces.
Under "begin", you can specify a subject line, under "opening" the opening
of your letter, and under "closing", a greeting. If you don't want anything
there, remove what's inside the braces but keep the declaration.

As in any latex document,the date is set automatically, but you can remove
it by typing change it by typing \texttt{\\date\{\}} in
the preamble (see point 1), or change it by typing \texttt{
date\{1/1/2009\}}.

3. If you wish to change tha package options (for example, if you like the
address to the left, or wich to print on actual letterhead without printing
the logo - see above), the package options are set in scientific word as
follows: \newline
a. choose the menu \emph{typeset/options} and packages\newline
b. select "\emph{vandyletter}" and click on "\emph{go native}" (don't click
the "modify" button: it wont' work)\newline
c. add the option names separated by commas in square brackets before
\{vandyletter\}, e.g.\newline
\texttt{[color-logo,addressright]\{vandyletter\}}

4. Finally, compile the file using pdflatex, that is, use the "pdf" button
to create a pdf directly.

Enjoy! -- Andrea

\closing{
Sincerely,
}

\end{letter}

\end{document}
